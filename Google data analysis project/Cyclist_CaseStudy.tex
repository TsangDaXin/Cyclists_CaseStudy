% Options for packages loaded elsewhere
\PassOptionsToPackage{unicode}{hyperref}
\PassOptionsToPackage{hyphens}{url}
%
\documentclass[
]{article}
\usepackage{amsmath,amssymb}
\usepackage{iftex}
\ifPDFTeX
  \usepackage[T1]{fontenc}
  \usepackage[utf8]{inputenc}
  \usepackage{textcomp} % provide euro and other symbols
\else % if luatex or xetex
  \usepackage{unicode-math} % this also loads fontspec
  \defaultfontfeatures{Scale=MatchLowercase}
  \defaultfontfeatures[\rmfamily]{Ligatures=TeX,Scale=1}
\fi
\usepackage{lmodern}
\ifPDFTeX\else
  % xetex/luatex font selection
\fi
% Use upquote if available, for straight quotes in verbatim environments
\IfFileExists{upquote.sty}{\usepackage{upquote}}{}
\IfFileExists{microtype.sty}{% use microtype if available
  \usepackage[]{microtype}
  \UseMicrotypeSet[protrusion]{basicmath} % disable protrusion for tt fonts
}{}
\makeatletter
\@ifundefined{KOMAClassName}{% if non-KOMA class
  \IfFileExists{parskip.sty}{%
    \usepackage{parskip}
  }{% else
    \setlength{\parindent}{0pt}
    \setlength{\parskip}{6pt plus 2pt minus 1pt}}
}{% if KOMA class
  \KOMAoptions{parskip=half}}
\makeatother
\usepackage{xcolor}
\usepackage[margin=1in]{geometry}
\usepackage{color}
\usepackage{fancyvrb}
\newcommand{\VerbBar}{|}
\newcommand{\VERB}{\Verb[commandchars=\\\{\}]}
\DefineVerbatimEnvironment{Highlighting}{Verbatim}{commandchars=\\\{\}}
% Add ',fontsize=\small' for more characters per line
\usepackage{framed}
\definecolor{shadecolor}{RGB}{248,248,248}
\newenvironment{Shaded}{\begin{snugshade}}{\end{snugshade}}
\newcommand{\AlertTok}[1]{\textcolor[rgb]{0.94,0.16,0.16}{#1}}
\newcommand{\AnnotationTok}[1]{\textcolor[rgb]{0.56,0.35,0.01}{\textbf{\textit{#1}}}}
\newcommand{\AttributeTok}[1]{\textcolor[rgb]{0.13,0.29,0.53}{#1}}
\newcommand{\BaseNTok}[1]{\textcolor[rgb]{0.00,0.00,0.81}{#1}}
\newcommand{\BuiltInTok}[1]{#1}
\newcommand{\CharTok}[1]{\textcolor[rgb]{0.31,0.60,0.02}{#1}}
\newcommand{\CommentTok}[1]{\textcolor[rgb]{0.56,0.35,0.01}{\textit{#1}}}
\newcommand{\CommentVarTok}[1]{\textcolor[rgb]{0.56,0.35,0.01}{\textbf{\textit{#1}}}}
\newcommand{\ConstantTok}[1]{\textcolor[rgb]{0.56,0.35,0.01}{#1}}
\newcommand{\ControlFlowTok}[1]{\textcolor[rgb]{0.13,0.29,0.53}{\textbf{#1}}}
\newcommand{\DataTypeTok}[1]{\textcolor[rgb]{0.13,0.29,0.53}{#1}}
\newcommand{\DecValTok}[1]{\textcolor[rgb]{0.00,0.00,0.81}{#1}}
\newcommand{\DocumentationTok}[1]{\textcolor[rgb]{0.56,0.35,0.01}{\textbf{\textit{#1}}}}
\newcommand{\ErrorTok}[1]{\textcolor[rgb]{0.64,0.00,0.00}{\textbf{#1}}}
\newcommand{\ExtensionTok}[1]{#1}
\newcommand{\FloatTok}[1]{\textcolor[rgb]{0.00,0.00,0.81}{#1}}
\newcommand{\FunctionTok}[1]{\textcolor[rgb]{0.13,0.29,0.53}{\textbf{#1}}}
\newcommand{\ImportTok}[1]{#1}
\newcommand{\InformationTok}[1]{\textcolor[rgb]{0.56,0.35,0.01}{\textbf{\textit{#1}}}}
\newcommand{\KeywordTok}[1]{\textcolor[rgb]{0.13,0.29,0.53}{\textbf{#1}}}
\newcommand{\NormalTok}[1]{#1}
\newcommand{\OperatorTok}[1]{\textcolor[rgb]{0.81,0.36,0.00}{\textbf{#1}}}
\newcommand{\OtherTok}[1]{\textcolor[rgb]{0.56,0.35,0.01}{#1}}
\newcommand{\PreprocessorTok}[1]{\textcolor[rgb]{0.56,0.35,0.01}{\textit{#1}}}
\newcommand{\RegionMarkerTok}[1]{#1}
\newcommand{\SpecialCharTok}[1]{\textcolor[rgb]{0.81,0.36,0.00}{\textbf{#1}}}
\newcommand{\SpecialStringTok}[1]{\textcolor[rgb]{0.31,0.60,0.02}{#1}}
\newcommand{\StringTok}[1]{\textcolor[rgb]{0.31,0.60,0.02}{#1}}
\newcommand{\VariableTok}[1]{\textcolor[rgb]{0.00,0.00,0.00}{#1}}
\newcommand{\VerbatimStringTok}[1]{\textcolor[rgb]{0.31,0.60,0.02}{#1}}
\newcommand{\WarningTok}[1]{\textcolor[rgb]{0.56,0.35,0.01}{\textbf{\textit{#1}}}}
\usepackage{graphicx}
\makeatletter
\def\maxwidth{\ifdim\Gin@nat@width>\linewidth\linewidth\else\Gin@nat@width\fi}
\def\maxheight{\ifdim\Gin@nat@height>\textheight\textheight\else\Gin@nat@height\fi}
\makeatother
% Scale images if necessary, so that they will not overflow the page
% margins by default, and it is still possible to overwrite the defaults
% using explicit options in \includegraphics[width, height, ...]{}
\setkeys{Gin}{width=\maxwidth,height=\maxheight,keepaspectratio}
% Set default figure placement to htbp
\makeatletter
\def\fps@figure{htbp}
\makeatother
\setlength{\emergencystretch}{3em} % prevent overfull lines
\providecommand{\tightlist}{%
  \setlength{\itemsep}{0pt}\setlength{\parskip}{0pt}}
\setcounter{secnumdepth}{-\maxdimen} % remove section numbering
\ifLuaTeX
  \usepackage{selnolig}  % disable illegal ligatures
\fi
\usepackage{bookmark}
\IfFileExists{xurl.sty}{\usepackage{xurl}}{} % add URL line breaks if available
\urlstyle{same}
\hypersetup{
  pdftitle={Cyclists\_CaseStudy},
  pdfauthor={Tsang Da Xin},
  hidelinks,
  pdfcreator={LaTeX via pandoc}}

\title{Cyclists\_CaseStudy}
\author{Tsang Da Xin}
\date{2024-09-01}

\begin{document}
\maketitle

\section{Prepare stage}\label{prepare-stage}

R program is used due each excel files contain large amount of data and
visualizations needed to complete the analysis

\subsection{Step 1}\label{step-1}

Load the library of the packages you have installed

\begin{Shaded}
\begin{Highlighting}[]
\FunctionTok{library}\NormalTok{(tidyverse)}
\end{Highlighting}
\end{Shaded}

\begin{verbatim}
## -- Attaching core tidyverse packages ------------------------ tidyverse 2.0.0 --
## v dplyr     1.1.4     v readr     2.1.5
## v forcats   1.0.0     v stringr   1.5.1
## v ggplot2   3.5.1     v tibble    3.2.1
## v lubridate 1.9.3     v tidyr     1.3.1
## v purrr     1.0.2     
## -- Conflicts ------------------------------------------ tidyverse_conflicts() --
## x dplyr::filter() masks stats::filter()
## x dplyr::lag()    masks stats::lag()
## i Use the conflicted package (<http://conflicted.r-lib.org/>) to force all conflicts to become errors
\end{verbatim}

\begin{Shaded}
\begin{Highlighting}[]
\FunctionTok{library}\NormalTok{(readr)}
\FunctionTok{library}\NormalTok{(lubridate)}
\FunctionTok{library}\NormalTok{(ggplot2)}
\FunctionTok{library}\NormalTok{(dplyr)}
\FunctionTok{library}\NormalTok{(sf)}
\end{Highlighting}
\end{Shaded}

\begin{verbatim}
## Linking to GEOS 3.12.1, GDAL 3.8.4, PROJ 9.3.1; sf_use_s2() is TRUE
\end{verbatim}

\begin{Shaded}
\begin{Highlighting}[]
\FunctionTok{library}\NormalTok{(scales)}
\end{Highlighting}
\end{Shaded}

\begin{verbatim}
## 
## Attaching package: 'scales'
## 
## The following object is masked from 'package:purrr':
## 
##     discard
## 
## The following object is masked from 'package:readr':
## 
##     col_factor
\end{verbatim}

\begin{Shaded}
\begin{Highlighting}[]
\FunctionTok{library}\NormalTok{(readxl)}
\end{Highlighting}
\end{Shaded}

Then , we combine all the datasets (January - December)

\begin{Shaded}
\begin{Highlighting}[]
\NormalTok{Jan\_tripdata }\OtherTok{\textless{}{-}} \FunctionTok{read\_excel}\NormalTok{(}\StringTok{"C:/Users/60122/OneDrive/Desktop/Google data analysis project/testing/202101{-}divvy{-}tripdata.xlsx"}\NormalTok{)}

\NormalTok{Feb\_tripdata }\OtherTok{\textless{}{-}} \FunctionTok{read\_excel}\NormalTok{(}\StringTok{"C:/Users/60122/OneDrive/Desktop/Google data analysis project/testing/202102{-}divvy{-}tripdata.xlsx"}\NormalTok{)}

\NormalTok{March\_tripdata }\OtherTok{\textless{}{-}} \FunctionTok{read\_excel}\NormalTok{(}\StringTok{"C:/Users/60122/OneDrive/Desktop/Google data analysis project/testing/202103{-}divvy{-}tripdata.xlsx"}\NormalTok{)}

\NormalTok{April\_tripdata }\OtherTok{\textless{}{-}} \FunctionTok{read\_excel}\NormalTok{(}\StringTok{"C:/Users/60122/OneDrive/Desktop/Google data analysis project/testing/202104{-}divvy{-}tripdata.xlsx"}\NormalTok{)}

\NormalTok{May\_tripdata }\OtherTok{\textless{}{-}} \FunctionTok{read\_excel}\NormalTok{(}\StringTok{"C:/Users/60122/OneDrive/Desktop/Google data analysis project/testing/202105{-}divvy{-}tripdata.xlsx"}\NormalTok{)}

\NormalTok{June\_tripdata }\OtherTok{\textless{}{-}} \FunctionTok{read\_excel}\NormalTok{(}\StringTok{"C:/Users/60122/OneDrive/Desktop/Google data analysis project/testing/202106{-}divvy{-}tripdata.xlsx"}\NormalTok{)}

\NormalTok{July\_tripdata }\OtherTok{\textless{}{-}} \FunctionTok{read\_excel}\NormalTok{(}\StringTok{"C:/Users/60122/OneDrive/Desktop/Google data analysis project/testing/202107{-}divvy{-}tripdata.xlsx"}\NormalTok{)}

\NormalTok{Aug\_tripdata }\OtherTok{\textless{}{-}} \FunctionTok{read\_excel}\NormalTok{(}\StringTok{"C:/Users/60122/OneDrive/Desktop/Google data analysis project/testing/202108{-}divvy{-}tripdata.xlsx"}\NormalTok{)}

\NormalTok{Sep\_tripdata }\OtherTok{\textless{}{-}} \FunctionTok{read\_excel}\NormalTok{(}\StringTok{"C:/Users/60122/OneDrive/Desktop/Google data analysis project/testing/202109{-}divvy{-}tripdata.xlsx"}\NormalTok{)}

\NormalTok{Oct\_tripdata }\OtherTok{\textless{}{-}} \FunctionTok{read\_excel}\NormalTok{(}\StringTok{"C:/Users/60122/OneDrive/Desktop/Google data analysis project/testing/202110{-}divvy{-}tripdata.xlsx"}\NormalTok{)}

\NormalTok{Nov\_tripdata }\OtherTok{\textless{}{-}} \FunctionTok{read\_excel}\NormalTok{(}\StringTok{"C:/Users/60122/OneDrive/Desktop/Google data analysis project/testing/202111{-}divvy{-}tripdata.xlsx"}\NormalTok{)}

\NormalTok{Dec\_tripdata }\OtherTok{\textless{}{-}} \FunctionTok{read\_excel}\NormalTok{(}\StringTok{"C:/Users/60122/OneDrive/Desktop/Google data analysis project/testing/202112{-}divvy{-}tripdata.xlsx"}\NormalTok{)}
\end{Highlighting}
\end{Shaded}

Combine all monthly tables into one dataset using rbind function

\begin{Shaded}
\begin{Highlighting}[]
\NormalTok{totaltrips }\OtherTok{\textless{}{-}} \FunctionTok{rbind}\NormalTok{(Jan\_tripdata , Feb\_tripdata , March\_tripdata , April\_tripdata , May\_tripdata , June\_tripdata , July\_tripdata , Aug\_tripdata , Sep\_tripdata , Oct\_tripdata , Nov\_tripdata , Dec\_tripdata)}
\end{Highlighting}
\end{Shaded}

Then we use several function to inspect the columns for data integrity

\begin{Shaded}
\begin{Highlighting}[]
\FunctionTok{head}\NormalTok{(totaltrips)}
\end{Highlighting}
\end{Shaded}

\begin{verbatim}
## # A tibble: 6 x 13
##   ride_id          rideable_type started_at          ended_at           
##   <chr>            <chr>         <dttm>              <dttm>             
## 1 E19E6F1B8D4C42ED electric_bike 2021-01-23 16:14:19 2021-01-23 16:24:44
## 2 DC88F20C2C55F27F electric_bike 2021-01-27 18:43:08 2021-01-27 18:47:12
## 3 EC45C94683FE3F27 electric_bike 2021-01-21 22:35:54 2021-01-21 22:37:14
## 4 4FA453A75AE377DB electric_bike 2021-01-07 13:31:13 2021-01-07 13:42:55
## 5 BE5E8EB4E7263A0B electric_bike 2021-01-23 02:24:02 2021-01-23 02:24:45
## 6 5D8969F88C773979 electric_bike 2021-01-09 14:24:07 2021-01-09 15:17:54
## # i 9 more variables: start_station_name <chr>, start_station_id <chr>,
## #   end_station_name <chr>, end_station_id <chr>, start_lat <dbl>,
## #   start_lng <dbl>, end_lat <dbl>, end_lng <dbl>, member_casual <chr>
\end{verbatim}

\begin{Shaded}
\begin{Highlighting}[]
\FunctionTok{str}\NormalTok{(totaltrips)}
\end{Highlighting}
\end{Shaded}

\begin{verbatim}
## tibble [5,595,063 x 13] (S3: tbl_df/tbl/data.frame)
##  $ ride_id           : chr [1:5595063] "E19E6F1B8D4C42ED" "DC88F20C2C55F27F" "EC45C94683FE3F27" "4FA453A75AE377DB" ...
##  $ rideable_type     : chr [1:5595063] "electric_bike" "electric_bike" "electric_bike" "electric_bike" ...
##  $ started_at        : POSIXct[1:5595063], format: "2021-01-23 16:14:19" "2021-01-27 18:43:08" ...
##  $ ended_at          : POSIXct[1:5595063], format: "2021-01-23 16:24:44" "2021-01-27 18:47:12" ...
##  $ start_station_name: chr [1:5595063] "California Ave & Cortez St" "California Ave & Cortez St" "California Ave & Cortez St" "California Ave & Cortez St" ...
##  $ start_station_id  : chr [1:5595063] "17660" "17660" "17660" "17660" ...
##  $ end_station_name  : chr [1:5595063] NA NA NA NA ...
##  $ end_station_id    : chr [1:5595063] NA NA NA NA ...
##  $ start_lat         : num [1:5595063] 41.9 41.9 41.9 41.9 41.9 ...
##  $ start_lng         : num [1:5595063] -87.7 -87.7 -87.7 -87.7 -87.7 ...
##  $ end_lat           : num [1:5595063] 41.9 41.9 41.9 41.9 41.9 ...
##  $ end_lng           : num [1:5595063] -87.7 -87.7 -87.7 -87.7 -87.7 ...
##  $ member_casual     : chr [1:5595063] "member" "member" "member" "member" ...
\end{verbatim}

\begin{Shaded}
\begin{Highlighting}[]
\FunctionTok{colnames}\NormalTok{(totaltrips)}
\end{Highlighting}
\end{Shaded}

\begin{verbatim}
##  [1] "ride_id"            "rideable_type"      "started_at"        
##  [4] "ended_at"           "start_station_name" "start_station_id"  
##  [7] "end_station_name"   "end_station_id"     "start_lat"         
## [10] "start_lng"          "end_lat"            "end_lng"           
## [13] "member_casual"
\end{verbatim}

\#\#Step 2

First , we remove the irrevalant columns that won't be used

\begin{Shaded}
\begin{Highlighting}[]
\NormalTok{totaltrips }\OtherTok{\textless{}{-}}\NormalTok{ totaltrips }\SpecialCharTok{\%\textgreater{}\%} \FunctionTok{select}\NormalTok{(}\SpecialCharTok{{-}}\FunctionTok{c}\NormalTok{(start\_lat,start\_lng,end\_lat,end\_lng,start\_station\_id,end\_station\_id,end\_station\_name))}
\end{Highlighting}
\end{Shaded}

Then we clean the datasets with null values

\begin{Shaded}
\begin{Highlighting}[]
\NormalTok{totaltrips }\OtherTok{\textless{}{-}} \FunctionTok{na.omit}\NormalTok{(totaltrips) }\SpecialCharTok{\%\textgreater{}\%} \FunctionTok{distinct}\NormalTok{()}
\end{Highlighting}
\end{Shaded}

After removing row of data with null values , I have decided to remove
any rows that contain ``docked\_bike'' , as it does not provide a solid
visualization to the problem statement . Thus , I classify them as dirty
data.

\begin{Shaded}
\begin{Highlighting}[]
\NormalTok{rideable\_type }\OtherTok{\textless{}{-}}\NormalTok{ totaltrips}\SpecialCharTok{$}\NormalTok{rideable\_type}
\NormalTok{totaltrips }\OtherTok{\textless{}{-}} \FunctionTok{subset}\NormalTok{(totaltrips , rideable\_type }\SpecialCharTok{!=} \StringTok{"docked\_bike"}\NormalTok{)}
\end{Highlighting}
\end{Shaded}

Again , we review the datasets to check for consistency

Next, we added new columns ``ride\_length'' , ``day\_of\_week'' and
``month'' .

\begin{Shaded}
\begin{Highlighting}[]
\NormalTok{totaltrips }\OtherTok{\textless{}{-}}\NormalTok{ totaltrips }\SpecialCharTok{\%\textgreater{}\%}                                                                                            \FunctionTok{mutate}\NormalTok{(}\AttributeTok{ride\_length =} \FunctionTok{difftime}\NormalTok{(ended\_at,started\_at,}\AttributeTok{units=}\StringTok{"mins"}\NormalTok{)) }\SpecialCharTok{\%\textgreater{}\%}                                                  \FunctionTok{mutate}\NormalTok{(}\AttributeTok{day\_of\_week =} \FunctionTok{wday}\NormalTok{(totaltrips}\SpecialCharTok{$}\NormalTok{started\_at)) }\SpecialCharTok{\%\textgreater{}\%}                                                                 \FunctionTok{mutate}\NormalTok{(}\AttributeTok{month =} \FunctionTok{format}\NormalTok{(}\FunctionTok{as.Date}\NormalTok{(totaltrips}\SpecialCharTok{$}\NormalTok{started\_at,}\AttributeTok{format=}\StringTok{"\%d/\%m/\%Y"}\NormalTok{),}\StringTok{"\%m"}\NormalTok{))}
\end{Highlighting}
\end{Shaded}

\subsection{Step 3 - Analyze}\label{step-3---analyze}

Now , its time to analyze our data. But before we do that , lets create
some additional variables for ease of calculations in the future

\begin{Shaded}
\begin{Highlighting}[]
\NormalTok{ride\_length }\OtherTok{\textless{}{-}}\NormalTok{ totaltrips}\SpecialCharTok{$}\NormalTok{ride\_length}
\NormalTok{member\_casual }\OtherTok{\textless{}{-}}\NormalTok{ totaltrips}\SpecialCharTok{$}\NormalTok{member\_casual}
\NormalTok{day\_of\_week }\OtherTok{\textless{}{-}}\NormalTok{ totaltrips}\SpecialCharTok{$}\NormalTok{day\_of\_week}
\NormalTok{rideable\_type }\OtherTok{\textless{}{-}}\NormalTok{ totaltrips}\SpecialCharTok{$}\NormalTok{rideable\_type}
\end{Highlighting}
\end{Shaded}

The code chunks below shows the step on calculating average ride length
for each month of the year

\begin{Shaded}
\begin{Highlighting}[]
\NormalTok{totaltrips }\SpecialCharTok{\%\textgreater{}\%} \FunctionTok{group\_by}\NormalTok{(month,member\_casual) }\SpecialCharTok{\%\textgreater{}\%} \FunctionTok{summarize}\NormalTok{(}\AttributeTok{mean\_ridelength =} \FunctionTok{mean}\NormalTok{(ride\_length)) }\SpecialCharTok{\%\textgreater{}\%} \FunctionTok{arrange}\NormalTok{(member\_casual) }\SpecialCharTok{\%\textgreater{}\%} \FunctionTok{print}\NormalTok{(}\AttributeTok{n=}\DecValTok{24}\NormalTok{) }\CommentTok{\#display up to 24 rows of data}
\end{Highlighting}
\end{Shaded}

\begin{verbatim}
## `summarise()` has grouped output by 'month'. You can override using the
## `.groups` argument.
\end{verbatim}

\begin{verbatim}
## # A tibble: 24 x 3
## # Groups:   month [12]
##    month member_casual mean_ridelength
##    <chr> <chr>         <drtn>         
##  1 01    casual        20.62135 mins  
##  2 02    casual        35.11794 mins  
##  3 03    casual        28.89385 mins  
##  4 04    casual        28.77257 mins  
##  5 05    casual        29.54601 mins  
##  6 06    casual        28.11505 mins  
##  7 07    casual        26.48611 mins  
##  8 08    casual        25.69609 mins  
##  9 09    casual        24.77236 mins  
## 10 10    casual        23.09008 mins  
## 11 11    casual        19.67964 mins  
## 12 12    casual        19.86508 mins  
## 13 01    member        12.97413 mins  
## 14 02    member        18.41053 mins  
## 15 03    member        13.98886 mins  
## 16 04    member        14.70024 mins  
## 17 05    member        14.75050 mins  
## 18 06    member        14.69884 mins  
## 19 07    member        14.20671 mins  
## 20 08    member        13.99501 mins  
## 21 09    member        13.64451 mins  
## 22 10    member        12.60073 mins  
## 23 11    member        11.55059 mins  
## 24 12    member        11.12541 mins
\end{verbatim}

Then we went on to calculate average ride length for each day of the
week

\begin{Shaded}
\begin{Highlighting}[]
\NormalTok{totaltrips }\SpecialCharTok{\%\textgreater{}\%} \FunctionTok{group\_by}\NormalTok{(day\_of\_week, member\_casual) }\SpecialCharTok{\%\textgreater{}\%} \FunctionTok{summarize}\NormalTok{(}\AttributeTok{mean\_ridelength =} \FunctionTok{mean}\NormalTok{(ride\_length)) }\SpecialCharTok{\%\textgreater{}\%} \FunctionTok{arrange}\NormalTok{(member\_casual)}
\end{Highlighting}
\end{Shaded}

\begin{verbatim}
## `summarise()` has grouped output by 'day_of_week'. You can override using the
## `.groups` argument.
\end{verbatim}

\begin{verbatim}
## # A tibble: 14 x 3
## # Groups:   day_of_week [7]
##    day_of_week member_casual mean_ridelength
##          <dbl> <chr>         <drtn>         
##  1           1 casual        30.00925 mins  
##  2           2 casual        26.32965 mins  
##  3           3 casual        23.43790 mins  
##  4           4 casual        22.68285 mins  
##  5           5 casual        22.85220 mins  
##  6           6 casual        24.46281 mins  
##  7           7 casual        28.23401 mins  
##  8           1 member        15.80425 mins  
##  9           2 member        13.33517 mins  
## 10           3 member        12.84296 mins  
## 11           4 member        12.86155 mins  
## 12           5 member        12.83994 mins  
## 13           6 member        13.36594 mins  
## 14           7 member        15.40425 mins
\end{verbatim}

Next , we calculate total amount of rides by month

\begin{Shaded}
\begin{Highlighting}[]
\NormalTok{totaltrips }\SpecialCharTok{\%\textgreater{}\%} \FunctionTok{group\_by}\NormalTok{(month , member\_casual) }\SpecialCharTok{\%\textgreater{}\%} \FunctionTok{summarize}\NormalTok{(}\AttributeTok{total\_rides =} \FunctionTok{n}\NormalTok{()) }\SpecialCharTok{\%\textgreater{}\%} \FunctionTok{arrange}\NormalTok{(member\_casual) }\SpecialCharTok{\%\textgreater{}\%} \FunctionTok{print}\NormalTok{(}\AttributeTok{n=}\DecValTok{24}\NormalTok{)}
\end{Highlighting}
\end{Shaded}

\begin{verbatim}
## `summarise()` has grouped output by 'month'. You can override using the
## `.groups` argument.
\end{verbatim}

\begin{verbatim}
## # A tibble: 24 x 3
## # Groups:   month [12]
##    month member_casual total_rides
##    <chr> <chr>               <int>
##  1 01    casual              13811
##  2 02    casual               7948
##  3 03    casual              63211
##  4 04    casual             101467
##  5 05    casual             187825
##  6 06    casual             276118
##  7 07    casual             336774
##  8 08    casual             320321
##  9 09    casual             280825
## 10 10    casual             188425
## 11 11    casual              74264
## 12 12    casual              47988
## 13 01    member              72292
## 14 02    member              36357
## 15 03    member             134780
## 16 04    member             184993
## 17 05    member             246711
## 18 06    member             321668
## 19 07    member             340675
## 20 08    member             350508
## 21 09    member             346872
## 22 10    member             311707
## 23 11    member             202810
## 24 12    member             143561
\end{verbatim}

Calculate total ride per day of the week

\begin{Shaded}
\begin{Highlighting}[]
\NormalTok{totaltrips }\SpecialCharTok{\%\textgreater{}\%} \FunctionTok{group\_by}\NormalTok{ (day\_of\_week , member\_casual) }\SpecialCharTok{\%\textgreater{}\%} \FunctionTok{summarize}\NormalTok{(}\AttributeTok{total\_rides =} \FunctionTok{n}\NormalTok{()) }\SpecialCharTok{\%\textgreater{}\%} \FunctionTok{arrange}\NormalTok{(member\_casual)}
\end{Highlighting}
\end{Shaded}

\begin{verbatim}
## `summarise()` has grouped output by 'day_of_week'. You can override using the
## `.groups` argument.
\end{verbatim}

\begin{verbatim}
## # A tibble: 14 x 3
## # Groups:   day_of_week [7]
##    day_of_week member_casual total_rides
##          <dbl> <chr>               <int>
##  1           1 casual             359838
##  2           2 casual             211787
##  3           3 casual             205326
##  4           4 casual             211174
##  5           5 casual             217421
##  6           6 casual             272771
##  7           7 casual             420660
##  8           1 member             329961
##  9           2 member             366336
## 10           3 member             410610
## 11           4 member             420896
## 12           5 member             396137
## 13           6 member             389488
## 14           7 member             379506
\end{verbatim}

We also want to calculate the max , min and median

\begin{Shaded}
\begin{Highlighting}[]
\CommentTok{\#in order to get summary statistic of data by group}
\FunctionTok{aggregate}\NormalTok{(totaltrips}\SpecialCharTok{$}\NormalTok{ride\_length }\SpecialCharTok{\textasciitilde{}}\NormalTok{ totaltrips}\SpecialCharTok{$}\NormalTok{member\_casual, }\AttributeTok{FUN =}\NormalTok{ max)}
\end{Highlighting}
\end{Shaded}

\begin{verbatim}
##   totaltrips$member_casual totaltrips$ride_length
## 1                   casual          1559.933 mins
## 2                   member          1559.933 mins
\end{verbatim}

\begin{Shaded}
\begin{Highlighting}[]
\FunctionTok{aggregate}\NormalTok{(totaltrips}\SpecialCharTok{$}\NormalTok{ride\_length }\SpecialCharTok{\textasciitilde{}}\NormalTok{ totaltrips}\SpecialCharTok{$}\NormalTok{member\_casual, }\AttributeTok{FUN =}\NormalTok{ min)}
\end{Highlighting}
\end{Shaded}

\begin{verbatim}
##   totaltrips$member_casual totaltrips$ride_length
## 1                   casual         -58.03333 mins
## 2                   member         -54.08333 mins
\end{verbatim}

\begin{Shaded}
\begin{Highlighting}[]
\FunctionTok{aggregate}\NormalTok{(totaltrips}\SpecialCharTok{$}\NormalTok{ride\_length }\SpecialCharTok{\textasciitilde{}}\NormalTok{ totaltrips}\SpecialCharTok{$}\NormalTok{member\_casual, }\AttributeTok{FUN =}\NormalTok{ median)}
\end{Highlighting}
\end{Shaded}

\begin{verbatim}
##   totaltrips$member_casual totaltrips$ride_length
## 1                   casual              15.0 mins
## 2                   member               9.8 mins
\end{verbatim}

\#Share Now , lets visualize the data we have just analyze . First
visualization that we are performing is ``total ride per month'' and
``total rides per week''

\begin{Shaded}
\begin{Highlighting}[]
\FunctionTok{ggplot}\NormalTok{(}\AttributeTok{data =}\NormalTok{ totaltrips) }\SpecialCharTok{+} \FunctionTok{geom\_bar}\NormalTok{(}\AttributeTok{mapping =} \FunctionTok{aes}\NormalTok{(}\AttributeTok{x =}\NormalTok{ month , }\AttributeTok{fill =}\NormalTok{ member\_casual), }\AttributeTok{position =} \StringTok{"dodge"}\NormalTok{) }\SpecialCharTok{+} \FunctionTok{scale\_y\_continuous}\NormalTok{(}\AttributeTok{labels=}\NormalTok{comma) }\SpecialCharTok{+} \FunctionTok{ggtitle}\NormalTok{(}\StringTok{"Total month rides"}\NormalTok{) }\SpecialCharTok{+} \FunctionTok{labs}\NormalTok{(}\AttributeTok{x =} \StringTok{"month"}\NormalTok{ , }\AttributeTok{y=}\StringTok{"Rides"}\NormalTok{ , }\AttributeTok{fill=}\StringTok{"Membership Status"}\NormalTok{ , }\AttributeTok{subtitle =} \FunctionTok{paste0}\NormalTok{(}\StringTok{"January to December (2021)"}\NormalTok{))}
\end{Highlighting}
\end{Shaded}

\includegraphics{Cyclist_CaseStudy_files/figure-latex/unnamed-chunk-16-1.pdf}

\begin{Shaded}
\begin{Highlighting}[]
\FunctionTok{ggplot}\NormalTok{(}\AttributeTok{data =}\NormalTok{ totaltrips) }\SpecialCharTok{+} \FunctionTok{geom\_bar}\NormalTok{(}\AttributeTok{mapping =} \FunctionTok{aes}\NormalTok{(}\AttributeTok{x =}\NormalTok{ day\_of\_week , }\AttributeTok{fill =}\NormalTok{ member\_casual), }\AttributeTok{position =} \StringTok{"dodge"}\NormalTok{) }\SpecialCharTok{+} \FunctionTok{scale\_y\_continuous}\NormalTok{(}\AttributeTok{labels=}\NormalTok{comma) }\SpecialCharTok{+} \FunctionTok{scale\_x\_continuous}\NormalTok{(}\AttributeTok{breaks=}\FunctionTok{seq}\NormalTok{(}\DecValTok{1}\NormalTok{,}\DecValTok{7}\NormalTok{,}\DecValTok{1}\NormalTok{))  }\SpecialCharTok{+} \FunctionTok{ggtitle}\NormalTok{(}\StringTok{"Total rides in a week"}\NormalTok{) }\SpecialCharTok{+} \FunctionTok{labs}\NormalTok{(}\AttributeTok{x =} \StringTok{"Weekday"}\NormalTok{ , }\AttributeTok{y=}\StringTok{"Rides"}\NormalTok{ , }\AttributeTok{fill=}\StringTok{"Membership Status"}\NormalTok{ , }\AttributeTok{subtitle =} \FunctionTok{paste0}\NormalTok{(}\StringTok{"Monday {-} Sunday (2021)"}\NormalTok{))}
\end{Highlighting}
\end{Shaded}

\includegraphics{Cyclist_CaseStudy_files/figure-latex/unnamed-chunk-17-1.pdf}

Second visualization we will be doing is ``average ride per month'' and
``average ride per week''

\begin{Shaded}
\begin{Highlighting}[]
\NormalTok{totaltrips }\SpecialCharTok{\%\textgreater{}\%} \FunctionTok{group\_by}\NormalTok{(month,member\_casual) }\SpecialCharTok{\%\textgreater{}\%} \FunctionTok{summarize}\NormalTok{(}\AttributeTok{mean\_ridelength =} \FunctionTok{mean}\NormalTok{(ride\_length)) }\SpecialCharTok{\%\textgreater{}\%} \FunctionTok{ggplot}\NormalTok{(}\FunctionTok{aes}\NormalTok{(}\AttributeTok{x=}\NormalTok{ month , }\AttributeTok{y =}\NormalTok{ mean\_ridelength , }\AttributeTok{fill =}\NormalTok{ member\_casual)) }\SpecialCharTok{+} \FunctionTok{geom\_col}\NormalTok{(}\AttributeTok{position=}\StringTok{"dodge2"}\NormalTok{)}\SpecialCharTok{+} \FunctionTok{ggtitle}\NormalTok{(}\StringTok{"Average monthly ride"}\NormalTok{) }\SpecialCharTok{+} \FunctionTok{labs}\NormalTok{(}\AttributeTok{x=}\StringTok{"Month"}\NormalTok{ , }\AttributeTok{y=}\StringTok{"Minutes"}\NormalTok{ , }\AttributeTok{fill=}\StringTok{"Membership status"}\NormalTok{)}
\end{Highlighting}
\end{Shaded}

\begin{verbatim}
## `summarise()` has grouped output by 'month'. You can override using the
## `.groups` argument.
## Don't know how to automatically pick scale for object of type <difftime>.
## Defaulting to continuous.
\end{verbatim}

\includegraphics{Cyclist_CaseStudy_files/figure-latex/unnamed-chunk-18-1.pdf}

\begin{Shaded}
\begin{Highlighting}[]
\NormalTok{totaltrips }\SpecialCharTok{\%\textgreater{}\%} \FunctionTok{group\_by}\NormalTok{(day\_of\_week,member\_casual) }\SpecialCharTok{\%\textgreater{}\%} \FunctionTok{summarize}\NormalTok{(}\AttributeTok{mean\_ridelength =} \FunctionTok{mean}\NormalTok{(ride\_length)) }\SpecialCharTok{\%\textgreater{}\%} \FunctionTok{ggplot}\NormalTok{(}\FunctionTok{aes}\NormalTok{(}\AttributeTok{x=}\NormalTok{ day\_of\_week , }\AttributeTok{y =}\NormalTok{ mean\_ridelength , }\AttributeTok{fill =}\NormalTok{ member\_casual)) }\SpecialCharTok{+} \FunctionTok{scale\_x\_continuous}\NormalTok{(}\AttributeTok{breaks =} \FunctionTok{seq}\NormalTok{(}\DecValTok{1}\NormalTok{,}\DecValTok{7}\NormalTok{,}\DecValTok{1}\NormalTok{)) }\SpecialCharTok{+} \FunctionTok{geom\_col}\NormalTok{(}\AttributeTok{position=}\StringTok{"dodge2"}\NormalTok{)}\SpecialCharTok{+} \FunctionTok{ggtitle}\NormalTok{(}\StringTok{"Average weekly ride"}\NormalTok{) }\SpecialCharTok{+} \FunctionTok{labs}\NormalTok{(}\AttributeTok{x=}\StringTok{"Weekdays"}\NormalTok{ , }\AttributeTok{y=}\StringTok{"Minutes"}\NormalTok{ , }\AttributeTok{fill=}\StringTok{"Membership status"}\NormalTok{ , }\AttributeTok{subtitle =} \FunctionTok{paste0}\NormalTok{(}\StringTok{"Monday {-} Sunday (2021)"}\NormalTok{) ) }
\end{Highlighting}
\end{Shaded}

\begin{verbatim}
## `summarise()` has grouped output by 'day_of_week'. You can override using the
## `.groups` argument.
## Don't know how to automatically pick scale for object of type <difftime>.
## Defaulting to continuous.
\end{verbatim}

\includegraphics{Cyclist_CaseStudy_files/figure-latex/unnamed-chunk-19-1.pdf}

It also essential to visualize the different type of bike riders as they
may have different bike riding patterns.

\begin{Shaded}
\begin{Highlighting}[]
\FunctionTok{ggplot}\NormalTok{(}\AttributeTok{data=}\NormalTok{ totaltrips) }\SpecialCharTok{+} \FunctionTok{geom\_bar}\NormalTok{(}\AttributeTok{mapping =} \FunctionTok{aes}\NormalTok{(}\AttributeTok{x =}\NormalTok{ month , }\AttributeTok{fill =}\NormalTok{ member\_casual) , }\AttributeTok{position =} \StringTok{"dodge"}\NormalTok{) }\SpecialCharTok{+} \FunctionTok{scale\_y\_continuous}\NormalTok{(}\AttributeTok{labels=}\NormalTok{comma) }\SpecialCharTok{+} \FunctionTok{ggtitle}\NormalTok{(}\StringTok{"Total monthly ride for each bike"}\NormalTok{) }\SpecialCharTok{+} \FunctionTok{facet\_wrap}\NormalTok{(}\SpecialCharTok{\textasciitilde{}}\NormalTok{rideable\_type) }\SpecialCharTok{+} \FunctionTok{labs}\NormalTok{(}\AttributeTok{x=}\StringTok{"Month"}\NormalTok{,}\AttributeTok{y=}\StringTok{"Rides"}\NormalTok{ , }\AttributeTok{fill =} \StringTok{"Membership Status"}\NormalTok{ , }\AttributeTok{subtitle =} \FunctionTok{paste0}\NormalTok{(}\StringTok{"January {-} December (2021)"}\NormalTok{))}
\end{Highlighting}
\end{Shaded}

\includegraphics{Cyclist_CaseStudy_files/figure-latex/unnamed-chunk-20-1.pdf}

\begin{Shaded}
\begin{Highlighting}[]
\FunctionTok{ggplot}\NormalTok{(}\AttributeTok{data=}\NormalTok{ totaltrips) }\SpecialCharTok{+} \FunctionTok{geom\_bar}\NormalTok{(}\AttributeTok{mapping =} \FunctionTok{aes}\NormalTok{(}\AttributeTok{x =}\NormalTok{ day\_of\_week , }\AttributeTok{fill =}\NormalTok{ member\_casual) , }\AttributeTok{position =} \StringTok{"dodge"}\NormalTok{) }\SpecialCharTok{+} \FunctionTok{scale\_y\_continuous}\NormalTok{(}\AttributeTok{labels=}\NormalTok{comma) }\SpecialCharTok{+} \FunctionTok{scale\_x\_continuous}\NormalTok{(}\AttributeTok{breaks =}\NormalTok{ (}\FunctionTok{seq}\NormalTok{(}\DecValTok{1}\NormalTok{,}\DecValTok{7}\NormalTok{,}\DecValTok{1}\NormalTok{))) }\SpecialCharTok{+} \FunctionTok{ggtitle}\NormalTok{(}\StringTok{"Total weekly ride for each bike"}\NormalTok{) }\SpecialCharTok{+} \FunctionTok{facet\_wrap}\NormalTok{(}\SpecialCharTok{\textasciitilde{}}\NormalTok{rideable\_type) }\SpecialCharTok{+} \FunctionTok{labs}\NormalTok{(}\AttributeTok{x=}\StringTok{"Weekdat"}\NormalTok{,}\AttributeTok{y=}\StringTok{"Rides"}\NormalTok{ , }\AttributeTok{fill =} \StringTok{"Membership Status"}\NormalTok{ , }\AttributeTok{subtitle =} \FunctionTok{paste0}\NormalTok{(}\StringTok{"Monday {-} Sunday (2021)"}\NormalTok{))}
\end{Highlighting}
\end{Shaded}

\includegraphics{Cyclist_CaseStudy_files/figure-latex/unnamed-chunk-21-1.pdf}

Last but not least , lets visualize the average ride length

\begin{Shaded}
\begin{Highlighting}[]
\NormalTok{totaltrips }\SpecialCharTok{\%\textgreater{}\%} \FunctionTok{group\_by}\NormalTok{(month,member\_casual, rideable\_type) }\SpecialCharTok{\%\textgreater{}\%} \FunctionTok{summarize}\NormalTok{(}\AttributeTok{mean\_ridelength =} \FunctionTok{mean}\NormalTok{(ride\_length)) }\SpecialCharTok{\%\textgreater{}\%} \FunctionTok{ggplot}\NormalTok{(}\AttributeTok{mapping =} \FunctionTok{aes}\NormalTok{(}\AttributeTok{x=}\NormalTok{month , }\AttributeTok{y =}\NormalTok{ mean\_ridelength, }\AttributeTok{fill =}\NormalTok{ member\_casual)) }\SpecialCharTok{+} \FunctionTok{geom\_col}\NormalTok{(}\AttributeTok{position =} \StringTok{"dodge2"}\NormalTok{) }\SpecialCharTok{+} \FunctionTok{ggtitle}\NormalTok{(}\StringTok{"Average monthly ride length"}\NormalTok{) }\SpecialCharTok{+} \FunctionTok{labs}\NormalTok{(}\AttributeTok{x=}\StringTok{"Month"}\NormalTok{ , }\AttributeTok{y =}\StringTok{"Minutes"}\NormalTok{ , }\AttributeTok{fill =} \StringTok{"Membership\_Status"}\NormalTok{) }\SpecialCharTok{+} \FunctionTok{facet\_wrap}\NormalTok{(}\SpecialCharTok{\textasciitilde{}}\NormalTok{rideable\_type)}
\end{Highlighting}
\end{Shaded}

\begin{verbatim}
## `summarise()` has grouped output by 'month', 'member_casual'. You can override
## using the `.groups` argument.
## Don't know how to automatically pick scale for object of type <difftime>.
## Defaulting to continuous.
\end{verbatim}

\includegraphics{Cyclist_CaseStudy_files/figure-latex/unnamed-chunk-22-1.pdf}

\begin{Shaded}
\begin{Highlighting}[]
\NormalTok{totaltrips }\SpecialCharTok{\%\textgreater{}\%} \FunctionTok{group\_by}\NormalTok{(day\_of\_week,member\_casual, rideable\_type) }\SpecialCharTok{\%\textgreater{}\%} \FunctionTok{summarize}\NormalTok{(}\AttributeTok{mean\_ridelength =} \FunctionTok{mean}\NormalTok{(ride\_length)) }\SpecialCharTok{\%\textgreater{}\%} \FunctionTok{ggplot}\NormalTok{(}\AttributeTok{mapping =} \FunctionTok{aes}\NormalTok{(}\AttributeTok{x=}\NormalTok{day\_of\_week , }\AttributeTok{y =}\NormalTok{ mean\_ridelength, }\AttributeTok{fill =}\NormalTok{ member\_casual)) }\SpecialCharTok{+} \FunctionTok{scale\_x\_continuous}\NormalTok{(}\AttributeTok{breaks =}  \FunctionTok{seq}\NormalTok{(}\DecValTok{1}\NormalTok{,}\DecValTok{7}\NormalTok{,}\DecValTok{1}\NormalTok{))    }\SpecialCharTok{+} \FunctionTok{geom\_col}\NormalTok{(}\AttributeTok{position =} \StringTok{"dodge2"}\NormalTok{) }\SpecialCharTok{+} \FunctionTok{ggtitle}\NormalTok{(}\StringTok{"Average weekly ride length"}\NormalTok{) }\SpecialCharTok{+} \FunctionTok{labs}\NormalTok{(}\AttributeTok{x=}\StringTok{"Weekday"}\NormalTok{ , }\AttributeTok{y =}\StringTok{"Minutes"}\NormalTok{ , }\AttributeTok{fill =} \StringTok{"Membership\_Status"}\NormalTok{) }\SpecialCharTok{+} \FunctionTok{facet\_wrap}\NormalTok{(}\SpecialCharTok{\textasciitilde{}}\NormalTok{rideable\_type)}
\end{Highlighting}
\end{Shaded}

\begin{verbatim}
## `summarise()` has grouped output by 'day_of_week', 'member_casual'. You can
## override using the `.groups` argument.
## Don't know how to automatically pick scale for object of type <difftime>.
## Defaulting to continuous.
\end{verbatim}

\includegraphics{Cyclist_CaseStudy_files/figure-latex/unnamed-chunk-23-1.pdf}

\end{document}
